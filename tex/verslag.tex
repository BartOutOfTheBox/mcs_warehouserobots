%%%%%%%%%%%%%%%%%%%%%%%%%%%%%%%%%%%%%%%%%
% Short Sectioned Assignment
% LaTeX Template
% Version 1.0 (5/5/12)
%
% This template has been downloaded from:
% http://www.LaTeXTemplates.com
%
% Original author:
% Frits Wenneker (http://www.howtotex.com)
%
% License:
% CC BY-NC-SA 3.0 (http://creativecommons.org/licenses/by-nc-sa/3.0/)
%
%%%%%%%%%%%%%%%%%%%%%%%%%%%%%%%%%%%%%%%%%

%----------------------------------------------------------------------------------------
%	PACKAGES AND OTHER DOCUMENT CONFIGURATIONS
%----------------------------------------------------------------------------------------

\documentclass[paper=a4, fontsize=11pt]{scrartcl} % A4 paper and 11pt font size

\usepackage[T1]{fontenc} % Use 8-bit encoding that has 256 glyphs
\usepackage{fourier} % Use the Adobe Utopia font for the document - comment this line to return to the LaTeX default
\usepackage[english]{babel} % English language/hyphenation
\usepackage{amsmath,amsfonts,amsthm} % Math packages

\usepackage{lipsum} % Used for inserting dummy 'Lorem ipsum' text into the template

\usepackage{sectsty} % Allows customizing section commands
\allsectionsfont{\centering \normalfont\scshape} % Make all sections centered, the default font and small caps

\usepackage{fancyhdr} % Custom headers and footers
\pagestyle{fancyplain} % Makes all pages in the document conform to the custom headers and footers
\fancyhead{} % No page header - if you want one, create it in the same way as the footers below
\fancyfoot[L]{} % Empty left footer
\fancyfoot[C]{} % Empty center footer
\fancyfoot[R]{\thepage} % Page numbering for right footer
\renewcommand{\headrulewidth}{0pt} % Remove header underlines
\renewcommand{\footrulewidth}{0pt} % Remove footer underlines
\setlength{\headheight}{13.6pt} % Customize the height of the header

\numberwithin{equation}{section} % Number equations within sections (i.e. 1.1, 1.2, 2.1, 2.2 instead of 1, 2, 3, 4)
\numberwithin{figure}{section} % Number figures within sections (i.e. 1.1, 1.2, 2.1, 2.2 instead of 1, 2, 3, 4)
\numberwithin{table}{section} % Number tables within sections (i.e. 1.1, 1.2, 2.1, 2.2 instead of 1, 2, 3, 4)

\setlength\parindent{0pt} % Removes all indentation from paragraphs - comment this line for an assignment with lots of text

%----------------------------------------------------------------------------------------
%	TITLE SECTION
%----------------------------------------------------------------------------------------

\newcommand{\horrule}[1]{\rule{\linewidth}{#1}} % Create horizontal rule command with 1 argument of height

\title{	
\normalfont \normalsize 
\textsc{Katholieke Universiteit Leuven} \\ [25pt] % Your university, school and/or department name(s)
\horrule{0.5pt} \\[0.4cm] % Thin top horizontal rule
\huge MCS project: Part 2 \\ % The assignment title
\horrule{2pt} \\[0.5cm] % Thick bottom horizontal rule
}

\author{Bart Verhoeven \& Dieter Jordens} % Your name

\date{\normalsize Januari 4, 2015} % Today's date or a custom date

\begin{document}

\maketitle % Print the title

%----------------------------------------------------------------------------------------
%	PROBLEM 1
%----------------------------------------------------------------------------------------

\section{Design decisions IDP}

Introduced predicate and function symbols:
\begin{itemize}
	\item \textbf{I\_on(p1,p2)} is true if pallet p1 is initially on pallet p2 (ie on(Start,p1,p2)),
	\item \textbf{C\_on(t,p1,p2)} is true if there is a reason for pallet p1 to be on pallet p2 at timestamp t+1. This is the case when p2 is on top of the stack, and the robot puts p1 down on the position of p2.
	\item \textbf{Cn\_on(t,p1,p2)} is true if there is a reason for pallet p1 not to be on pallet p2 at timestamp t+1. This happens when the robot picks up pallet p1.
	\item \textbf{I\_carried(p1)} is true if pallet p1 is initially carried by the robot (ie carried(Start,p1)),
	\item \textbf{C\_carried(t,p)} is true if there is a reason for pallet p to be carried by the robot at timestamp t+1. This is the case if the robot picks up this pallet at timestamp t.
	\item \textbf{Cn\_carried(t,p)} is true if there is a reason for pallet p to not be carried by the robot at time t+1. This happens when the robot puts down this pallet at timestamp t.
	\item \textbf{I\_robotposition(l)} is true if the robot is initially located at l (ie robotposition(Start)=l),
	\item \textbf{C\_robotposition(t,l)} is true if there is a reason for the robot to be on location l at time t+1. The robot will be on nextPosition(t+1) if he moves forwards. And on prevPosition(t+1) if he moves backwards.
	\item \textbf{Cn\_robotposition(t,l)} is true if there is a reason for the robot not to be on location l at the next timestamp. This is the case when the robot moves.
	\item \textbf{I\_position(p,l)} is true if pallet p is initially at location l. ie. position(Start,p)=l.
	\item \textbf{C\_position(t,p,l)} is true if there is a reason for pallet p to be on location l at time t+1. This reason can be being put down, and being carried by a moving robot.
	\item \textbf{Cn\_position(t,p,l)} is true if there is a reason for pallet p not to be on location l at time t+1. This reason can be being picked up, and being carried by a moving robot.
	\item \textbf{I\_facing(dir)} is true if the robot is initially facing direction dir.
	\item \textbf{C\_facing(Time,dir)} is true if the robot has reason to face direction dir at time Time+1. This reason is turning.
	\item \textbf{Cn\_facing(Time,dir)} is true if the robot has a reason not to face direction dir at time Time+1. This reason is turning.
	\item \textbf{above(t, p1, p2)} is true if and only if $\langle$the pallet p1 is located above pallet p2 at time t, but 0 or more pallets are allowed to be in between those two pallets$\rangle$,
	\item \textbf{heightOf(t, pallet): height} is $\langle$the height of this pallet at time t, therefore the amount of pallets beneath this pallet$\rangle$,
	\item \textbf{free(t, p)} means it is possible to pick up pallet p at timestamp t,
	\item \textbf{partial nextPosition(Time): location} denotes the next grid(xco,yco) the robot would be on if it were to move forwards,
	\item \textbf{partial prevPosition(Time): location} denotes the next grid(xco,yco) the robot would be on if it were to move backwards.
\end{itemize}
%------------------------------------------------
%----------------------------------------------------------------------------------------
%	PROBLEM 2
%----------------------------------------------------------------------------------------

\section{tijdsbesteding}
 \begin{itemize}
	\item IDP: Bart 20u + Dieter 0.5u
	\item NUsmv: Bart 1u + Dieter 15u
 \end{itemize}
\end{document}